\documentclass[a4paper, 11pt]{article}

\usepackage[british]{babel}
\usepackage[autostyle]{csquotes}

\begin{document}

\title{Forecasting Prices Using Stock Market Index Data}
\author{Student Number: 690065435}
\date{December 2022}
\maketitle

\section{Introduction}
Stock markets from across the world are tracked using indices that measure a section of the stock market, such as the Nasdaq Composite Index. Price forecasting is a very important task in the financial industry, as it can be used to guide strategies.

Historically, financial institutions have used discretionary methods to make investment decisions -- they rely on fundamentals and the judgement of analysts \cite{harvey2017man}. However, with the rise of big data and computational power, systematic methods have become increasingly popular -- institutions use rules-based strategies that are implemented by a computer and involve little to no human intervention \cite{harvey2017man}. Systematic methods enable decisions to be made quickly, which leading market makers and high-frequency trading firms such as Jane Street Capital and Hudson River Trading use to exploit arbitrage opportunities and maximise profits by trading at high volumes \cite{aldridge2013high}. However, these firms do not publicly disclose their strategies, which makes it difficult to understand how they make decisions.

In this project, we will investigate the following question: can we use regression models on stock market index data to forecast prices effectively? My initial hypothesis is that this is not possible, as the stock market is a complex system that is difficult to predict.

Previous studies have investigated the prediction of stock market trends with regression on moving averages \cite{dinesh2021prediction}, but they have focused on individual stocks whereas I am focusing on a stock market index. This is a potential snag, as the index is not a direct representation of the stocks it tracks. The efficient market hypothesis also suggests that this will be difficult, as it states that asset prices reflect all available information, yet we would only be predicting according to a subset of information \cite{fama1970efficient}. 

\section{Methodology and Dataset}

\subsection{About Stock Exchange Data Set}

\subsection{Data Cleaning}

\subsection{Data Exploration}

\subsection{Feature Engineering}

\subsection{Data Filtering}

\subsection{Regression Models}

\subsection{Models Evaluation}

\section{Results}
\subsection{Ridge Regression}

\subsection{LASSO Regression}

\subsection{LSTM Regression}

\section{Discussion}

\subsection{Limitations of the Study}
The efficient market hypothesis states that asset prices reflect all available information, which makes it difficult to get arbitrage opportunities. However, Renaissance Technologies, a hedge fund that uses systematic methods, serves as a perfect counter-example -- their Medallion Fund has achieved 66.07\% annualised returns since 1988 \cite{cornell2020medallion}. This suggests that it may be possible to achieve better results given access to big data and computational power, whereas in this study we relied on limited information.

\subsection{Future Work}

\bibliography{main}
\bibliographystyle{ieeetr.bst}

\end{document}